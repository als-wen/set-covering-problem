\documentclass[12pt]{article}
\usepackage[english]{babel}
\usepackage{amsmath, amssymb}
\usepackage{geometry}
\usepackage{fancyhdr}
\usepackage{longtable}
\usepackage{caption}

\geometry{margin=1in}
\pagestyle{fancy}
\fancyhf{}
\rhead{Set Covering Problem}
\lhead{Optimization Problem}
\rfoot{\thepage}

\title{Set Covering Problem\\\large PIA: Part II}
\author{Operations Research, 032\\\large Team 6: Alan Alejandro Vargas González, 2086183}
\date{\ May, 2025}

\begin{document}

\maketitle

\section{Part II: Codification and Visualization}

\subsection{Model Codification}
The model was implemented in Python using the \texttt{PuLP} library. The complete code is stored in the set-covering-repository. 

\begin{verbatim}
def solving(m, n, costs, cov):
    prob = pulp.LpProblem("Set_Covering_Problem", pulp.LpMinimize)

    x = [pulp.LpVariable(f"x_{j}", cat="Binary") for j in range(n)]

    # Objective function
    prob += pulp.lpSum([costs[j] * x[j] for j in range(n)])

    # Constraints
    for i in range(m):
        prob += pulp.lpSum([x[j] for j in cov[i]]) >= 1

    start = time.time()
    prob.solve()
    amount_time = time.time() - start

    subset = [j for j in range(n) if pulp.value(x[j]) == 1]
    total_cost = pulp.value(prob.objective)

    return subset, total_cost, amount_time
\end{verbatim}

\subsection{Data Structures}
The next data structures were used:
\begin{itemize}
    \item List \texttt{costs}: contains the cost of each subset.
    \item List \texttt{coverage} lists: contains, for each element, the subsets that cover it.
    \item Binary variable \texttt{x[j]}: is the \( j \) subset selected?
\end{itemize}

\subsection{Data Visualization}
Validation of:
\begin{itemize}
    \item All the costs are positive.
    \item Each row is cover by at least one subset.
    \item There aren't indexes out of range.
\end{itemize}

\begin{verbatim}
def validation(m, n, costs, cov):
    assert len(costs) == n, "Length of costs don't match n"
    assert len(cov) == m, "Length of coverage don't match m"
    for c in costs:
        assert c > 0, "Costs need to be positive"
    for i in range(m):
        assert len(cov[i]) > 0, f"Row {i} doesn't have coverage"
        for col in cov[i]:
            assert 0 <= col < n, f"Index out of range in coverage of row {i}"
\end{verbatim}

\subsection{Preliminary Analysis}

Testing with small data, such as the \texttt{ scp's} instances, obtaining consistent results with known values. I created a small example to verify the correct functionality of the program:

\begin{itemize}
    \item \textbf{Elements:} 3  
    \item \textbf{Subsets:} 4  
    \item \textbf{Costs:} $[1, 2, 3, 4]$  
    \item \textbf{Coverage:}  
    \begin{itemize}
        \item Row 0: covered by subsets $\{0, 1\}$  
        \item Row 1: covered by subsets $\{1, 2\}$  
        \item Row 2: covered by subsets $\{2, 3\}$  
    \end{itemize}
\end{itemize}

\noindent
\textbf{Results:}  
\begin{itemize}
    \item \textbf{Subsets selected:} $[0, 2]$  
    \item \textbf{Total cost:} 4  
    \item \textbf{Execution time:} 0.0784\,s  
\end{itemize}

\paragraph*{Why these subsets?}  
\begin{itemize}
    \item \textbf{Row 0} is covered by subsets \([0, 1]\) \(\rightarrow\) Covered by subset \(0\).
    \item \textbf{Row 1} is covered by subsets \([1, 2]\) \(\rightarrow\) Covered by subset \(2\).
    \item \textbf{Row 2} is covered by subsets \([2, 3]\) \(\rightarrow\) Covered by subset \(2\).
\end{itemize}
Therefore, the subsets \(0\) and \(2\) are enough to cover all rows.

\paragraph*{Total cost}  
The total cost is:
\[
\text{Total Cost} = \text{cost}[0] + \text{cost}[2] = 1 + 3 = 4.
\]

\paragraph*{Execution time}  
The solver took \textbf{0.1287s} to find the optimal solution.

\end{document}
