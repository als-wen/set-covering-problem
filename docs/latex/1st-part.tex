\documentclass[12pt]{article}
\usepackage[english]{babel}
\usepackage{amsmath, amssymb}
\usepackage{geometry}
\usepackage{fancyhdr}
\usepackage{longtable}
\usepackage{caption}
\usepackage{hyperref}

\geometry{margin=1in}
\pagestyle{fancy}
\fancyhf{}
\rhead{Set Covering Problem}
\lhead{Optimization Problem}
\rfoot{\thepage}

\title{Set Covering Problem\\\large PIA: Part I}
\author{Operations Research, 032\\\large Team 6: Alan Alejandro Vargas González, 2086183}
\date{\ May, 2025}

\begin{document}

\maketitle

\section{Part I: Formulation and Planning}

\subsection{Definition}
The \textbf{Set Covering Problem (SCP)} consists in selecting a subset with minimum cost in a collection of subsets that are part of a universal set of elements. Each subset has certain elements, and the objective is to cover all elements with the minimum number of subsets.

\subsection{Objectives}
\begin{itemize}
    \item Minimize the total cost of the subsets selected.
    \item Include all elements of the universal set.
\end{itemize}

\subsection{Applications}
This problem appears in multiple problems of Operations Research, such as:
\begin{itemize}
    \item Installation's location (hospitals, service stations, etc.).
    \item Covering of resources.
    \item Distribution and planning of routes.
\end{itemize}

\subsection{Model classification}
\begin{itemize}
    \item Type: Full Binary Linear Programming.
    \item Why? Decision variables are binary, the objective function and restrictions are linear.   
\end{itemize}

\subsection{Model elements}
\textbf{Parameters}:
\begin{itemize}
    \item $m$: number of elements.
    \item $n$: number of subsets.
    \item $c_j$: Cost of selecting subset $j$.
    \item $a_{ij}$: binary matrix that indicates if the subset $j$ has the element $i$.
\end{itemize}

\textbf{Decision variables}:
\[
x_j = \begin{cases}
1 & \text{if the $j$ subset is selected} \\
0 & \text{it isn't selected}
\end{cases}
\]

\textbf{Objective function}:
\[
\min \sum_{j=1}^{n} c_j x_j
\]

\textbf{Constraints}:
\[
\sum_{j=1}^{n} a_{ij} x_j \geq 1 \quad \forall i = 1, \dots, m
\]

\subsection{Initial Design for Implementation}
\begin{itemize}
    \item Programming language: Python
    \item Libraries: \texttt{PuLP}, \texttt{pandas}, \texttt{matplotlib}
    \item Code structure: data input, model definition, solution, and result visualization.
\end{itemize}

\subsection{Project Planning}
\begin{longtable}{|c|p{8cm}|c|c|}
\hline
\textbf{Week} & \textbf{Activity} & \textbf{Tools} \\
\hline
1 & Formulation and Planning & Overleaf \\
2 & Codification and Visualization & Python (PuLP) \\
3 & Testing and Optimization & Python (Pandas) \\
4 & Final Results & Python (matplotlib) \\
5 & Presentation & Canva \\
6 & Final report & GitHub, LaTeX \\
\hline
\end{longtable}

\begin{thebibliography}{9}
\bibitem{balas1972} 
Balas, E., \& Padberg, M. W. (1972). On the Set-Covering Problem. 
\textit{Operations Research}, 20(6), 1152–1161. 
\url{http://dx.doi.org/10.1287/opre.20.6.1152}
\end{thebibliography}

\end{document}
